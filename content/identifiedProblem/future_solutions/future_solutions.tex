Even though the following solutions are not implemented yet, it is still important to mention them in order to see if our system would still be useful even with these new solutions. They will mostly solve the network scalability and congestion problems. But transaction costs are extremely correlated with network congestion: when the network is busy, people tend to pay higher fees to get their operations processed with a reasonable delay. 
\subsection{Sharding}
Sharding is a solution proposed to solve the Ethereum scalability problem that could make the network less busy and thus reduce gas fees. Sharding is inspired by database Shard architecture and intends to change the network structure. Using all nodes of the network to validate each and every operation might not be essential. Sharding proposes to divide the network into multiple sections called "shards". Each shard processes then its own set of addresses and transactions and has its own independent state.
But this method implies a lot of challenges, especially on the ethereum blockchain. Splitting the network into shards implies splitting the consensus mechanism in a balanced manner, but using proof of work: it is difficult to assign a miner to a particular shard since the miner can always compute the hashes of whichever block he wants. This is not the case on a proof of work consense since there are validators that can already lock some funds on a particular shard.
Sharding is already implemented on other blockchains (PoS) such as Elrond. 
\subsection{Proof of stake consensus}
The eth2 upgrades also intend to solve the scalability problem by changing the consensus mechanism from proof of work to proof of stake. This change will result in a more efficient network, using less energy (no miners anymore) and it will also provide stronger support to implement the sharding mechanism described above. However, the change has already been studied for a while but still there isn't any date of deployment. The introduction of this new consensus mechanism will certainly reduce the gas price by increasing the throughput of the network (gas prices spike when the network is too busy) but these fees will still exist. Even by using proof of stake one still needs to run the code of smart contracts.