The term Layer 2 refers to a series of blockchain solutions built on a different network running on top of layer 1, the main Ethereum network, through smart contracts. They are designed to increase the scalability of layer 1 while still being able to benefit from the security and the existing architecture of the Mainnet. The main idea for these protocols is to perform transactions off-chain to achieve a faster confirmation time, lower fees and traffic reduction on the Mainnet. \cite{gudgeon_sok_2020, noauthor_layer_nodate}.

The aim of this section is to systematise the panel of layer 2 protocols that emerged in the past decade. Ultimately, we conclude with a proof of need to show that {\projectName} brings novelties not captured by the panel of existing layer 2 solutions. To this end, we provide a comparison of the main solutions in terms of structure, synchronisation with layer 1, scalability power and security. Unsolved challenges and expected future adoption of these protocols are also addressed.  

% Motivation
% Description
% Points fort et faibles
% Adoption marché (utilisation)
% Future 

\subsubsection{Sidechains}

% Motivation
As previously mentioned, Ethereum network is one of the most used blockchain on the market. Nevertheless, many blockchains exist and many more are to come. With every blockchain come blockchain-specific limitation, one may want reduce its exposure to limitations by diversifying its blockchain networks. As highlighted previously, a daily Mainnet interactions DeFi protocol unveil high operation costs to the end-user. Then it could be favourable to hold low transactions on a cheaper and less secure blockchain while the account balance and high transactions is saved on the Ethereum blockchain. This other blockchain would be called in this case a sidechain to Ethereum blockchain.

% Description
As the name indicates, sidechains are independent Ethereum-compatible blockchain that runs alongside to Ethereum Mainnet. Side chains can employ different ledger parameters, security mechanisms and consensus models to process transactions. Their frameworks are chosen carefully to efficiently exercise activities in relation to the main networks it operates in. A sidechain can be designed to operate integrally with any blockchain. In other any form of data from the blockchain is cross-chain transferable. Furthermore, this paralleling between siblings chains permit a multi layer communication channels with a delocalization of the assets.

As described above, having this EVM compatible chain allows to increase the scability of the layer 1. The sidechain smart contracts will manage infrastructure and distributed sidechains statuses, while the Mainnet smart contrats manage validation proof and account states. One has to note that many sidechains can be connected to the Mainnet

Any decentralized applications that is constrained by low security exposure but high transaction throughput would optimise its operation network by accumulating sidechains.

% Technical 
It is connected to a Mainnet by a two-way bridge, which consists in five simple steps :
\begin{itemize}
    \item Mainnet funds are locked;
    \item Funds are unlocked on the sidechain;
    \item Transaction occurs with those funds;
    \item When finished funds on sidechain are burnt;
    \item When burnt signaled is received, Mainnet funds are unlocked.
\end{itemize}
This makes each sidechains self-contained, thus making it impossible for a waterfall attack to occur. 

% Points fort et faibles
%faible
Can it be really considered as a Layer 2 solution since it does not rely on the security of the main chain. This can be considered as the main drawback, this leak of security compared to its main chain can have consequences on the overall project. Nevertheless, if ones ensure the same consensus models as the Mainnet, with similar security then the whole system could be considered as stable. Previous example of fraud committed by a quorum validators has been seen. Showing that the decentralization brought by the sidechains could be harmful.

To guarantee the two-way bridge connection, users can be prevented from withdrawing their funds. This liquidity fastening is a must to ensure the cross-chain protocol to but in terms of consumer security it is considered at risk.

%fort
In contrary, this permits a pre-synthetic sharding. For deeper explanation please refer to \cite{XXX}, but in a few words it does improve operational speed and increase the capacity of the network by extending the Mainnet with sidechains, the load data on the data is thus distributed .

Those siblings could also permits a pre-use of more \textit{sustainable} and innovative consensus models, thus earlier shifting the environmental impact within the Mainnet consensus.

For Ethereum specific sidechains, they are based on the Ethereum Virtual Machine meaning that the compatibility is ensured.


% Adoption marché (utilisation)
The Ethereum scalability improvement is mainly based on sidechains for the moment. The main utilisation for sidechains are application-specific transactions such as NFT art, DAO voting and stable transactions, including micro-transactions.

Polygon : 

% Future 
Sidechains that could operates with multiple blockchains at the same time are currently under development. The main principle would be to allow for a multiple-way bridge. This siblings interoperability would allow for faster and cheaper transactions, as the network load would be spread across every chains.


\subsubsection{State channels}
% Motivation
Every applications do not need similar security standards. Imagine a DeFi service needs to take micro-payment, having those latter all recorded on the Mainnet would be gas expensive. Instead, the service could incentive the user to lock a sufficient amount of fund in a pool, that will latter serve to pay the user service utilisation. This utilisation could be recorded off-chain and the reported final value recorded on a smart contract on the Ethereum blockchain that will latter transfer the necessary funds to the service holder.

% Description
To better understand the state channels Layer 2 solution, the definition of each words in the solution names help greatly.

A State, is computationally defined as the condition of the variable linked to. While a channel is a communication vector employed between users interested in the mentionned variable. From those definition, one understand state channel as a place for transaction, interactions and record that would later be published into the Mainnet. The channel is only accessible by authorized parties and should be a safe place.
% Technical
A smart contract between two or more parties is settle on the Blockchain. This contract is dynamically interactive with the two users input. For example, receiving input x and y will change the smart contract to 0. While if the inputs were x and z, the state will change to 1. For the state smart contract to work, users would have to lock ETH into the smart contract to act as liquidity pool. The blockchain state has multi-signature lock to ensure the security of capitals, complete consensus has to be reached.

% Points fort
Only the users invited in the channel have access to those off-chains data and the interactions are instantaneous. Considering this last point, every messages in the channel is digitally signed by the user that authorized it. No cheating and early leaving would affect the user-pool. The second user can publish the state channel of the latest signed results.

As you are not recording those off-chains data on the Mainnet, interactions have low fees, except for the final state. This last fee is however negligible in comparison to the computation that it would have been needed. The Mainnet is less congested with data.

It is also possible for the entering users to set-up end time conditions, this gives higher flexibility to the users.

% point faible
The limitation of channels make impossible for the users to send funds off-chain to non-participants. For a simple recurring payments the amount of capital would be know beforehand but in the case of more complex state transactions this amount could become very large. Not talking about margin protection like smart contract.

The current state of the channel should be allowed to be published on the blockchain at any time. Meaning that at least one user has to publish this final state to protect each individual party's interests until the channel is totally closed. It has to be noted that the off-chain transaction are meaningful as long as the Mainnet is working.

This solution gets rid of the initial goal of the blockchain to act as a fully private solution. The multi-signature procedure allow each users to know the other channel users identity.

% Adoption marché (utilisation)
State channels are under employed compared to their potential. Indeed, the judging smart contract set-up at the very beginning is the promise for each parties to be partially treated. Meaning that this prior setting is of high importance, but may be a heavy step for potential users.

% Future 
This latter judge mechanism must be flawless, give interesting properties and the initial constraints must be known. Such application with well know constraints would be low level service, payment channel, direct contract and unsecured transactions (i.e. the score of a soccer game). 

Actual research are on the standardisation of a judge contract and user-friendly mechanism.


\subsubsection{Rollups}

Rollups are yet another type of scaling solution. The core idea is to execute transactions outside layer 1 and store the resulting transaction data as proof back on layer 1. Correct execution on layer 2 is ensured by a smart contract in layer 1 using the transaction data. 

Since computations are moved off-chain, e.g. on layer 2, transactions can be processed at much higher rate. On layer 1, they are committed in batches using data compression mechanisms to reduce both cost and network effects on Ethereum. 

Hence, Rollups can be viewed as a combination between state channels and side chains since they can create general purpose applications through the EVM compatibility while still fully making use of the security mechanisms deployed on Ethereum.

To incentivise users to execute and verify transactions correctly, Rollup protocols require staking a bond as collateral. If a fraudulent action is detected, the system slashes the bonds of the users involved. 

We can further distinguish between different types of Rollups when it comes to proofs and security model. 
\begin{itemize}
    \item \textit{Optimistic Rollups} optimistically assume that transactions are valid by default. In other words, the system don't check the transactions. If they indeed are valid, no additional work needs to be done. Otherwise, users can report invalid transactions to the system which then enters a dispute resolution mode. In this state, the Rollup will execute a fraud proof where the suspicious transactions are verified on the main Ethereum chain. If the transaction was indeed invalid, the state is reverted and the party that submitted it is punished : their staked bond is slashed. It is also worthy mentioning that users who report transactions also need to stake a bond to avoid unnecessary fraud proof computation. For the system to work correctly, the dispute resolution mode should be an exceptional state. 
    
    \item \textit{Zero-Knowleadge (ZK) Rollups} leverage a cryptographic instrument called SNARK (a type of zero-knowledge proofs) to produce validity proofs. More specifically, to emit a valid transaction in the system, the user has to provide a validity proof. In other words, invalid transactions are rejected right away. It also allows the smart contract of the ZK Rollup to maintain the state of transactions on layer 2 only using these proofs instead of the whole transaction data, reducing both storage requirement and latency. 
    \item \textit{Validium}'s method is extremely similar to that of a ZK Rollup, with the exception that ZK Rollup data availability is on-chain, but Validium data availability is off-chain. Validium can achieve far higher throughput as a result of this, and its operator can deny any user the opportunity to move their funds without ZK Rollup's data availability guarantees.
\end{itemize}



\begin{table}[htbp]
\caption{Comparison Optimistic and ZK Rollups}
\begin{center}
\begin{tabular}{|c|p{3.5cm}|p{3.5cm}|}
\hline
\textbf{}&\multicolumn{2}{|c|}{\textbf{Rollup Type}} \\
\cline{2-3} 
\textbf{} & \textbf{\textit{Optimistic}}& \textbf{\textit{ZK}} \\
\hline
Pros 
& Anything you can do on Ethereum layer 1, you can do with Optimistic rollups as it's EVM and Solidity compatible. & Faster finality time since the state is instantly verified once the proofs are sent to the main chain. \\
& All transaction data is stored on the layer 1 chain, meaning it's secure and decentralized   & Not vulnerable to the economic attacks that Optimistic rollups can be vulnerable to. \\
&   & Secure and decentralized, since the data that is needed to recover the state is stored on the layer 1 chain \\

Cons 
& Long wait times for on-chain transaction due to potential fraud challenges. & Some don't have EVM support.  \\
& An operator can influence transaction ordering. & Validity proofs are intense to compute – not worth it for applications with little on-chain activity. \\
&  & An operator can influence transaction. ordering  \\
\hline
\end{tabular}
\label{tab:comparison:rollups}
\end{center}
\end{table}


\subsection{Existing Layer 2 Protocols}

\begin{table}[htbp]
\caption{Comparison Layer 2 Protocols}
\begin{center}
\begin{tabular}{|c|c|c|c|c|c|c|c|c}
\hline
%\textbf{}&\multicolumn{2}{|c|}{\textbf{Rollup Type}} \\
\cline{2-3} 
\textbf{} & \textbf{\textit{Type}}& \textbf{\textit{Notable partner projects}} & \textbf{\textit{Transaction Costs}}  & \textbf{\textit{Transaction Speed}} & \textbf{\textit{Transaction Costs}} & \textbf{\textit{Transaction Costs}} & \textbf{\textit{Smart contract support }} & \textbf{\textit{EVM Compatible}} \\
\hline
Optimism & Optimistic rollups  & Synthetics, Uniswap  & & & &X  & X \\
Arbitrum & Optimistic rollups  & uniswap sushi, Bangor Auger, Reddit  & & & &X & X\\

\hline
\end{tabular}
\label{tab:comparison:layer2}
\end{center}
\end{table}



\subsubsection{Optimistic rollups:  Optimism and Arbitrum}

In terms of Optimistic Rollups, \textit{Optimism} and \textit{Arbitrum} are considered to be the most popular options which already started their deployment on the Mainnet. If they share most of their functionalities, several differences are worth mentioning :
\begin{itemize}
    \item To tackle the long withdrawal problem, Optimism decided to partner up with MakerDao. Several tokens such as DAI can benefit from the collaboration. 
    \item Arbitrum developed a algorithm to narrow down the scope of disputes so that only a few transactions need to be reevaluated on layer 1 when a fraud is reported.
    \item Transaction ordering is handled differently in both systems. Arbirtrum uses a centralised sequencer while Optimism uses third parties for a given amount of time via an auction mechanism.
\end{itemize}


\subsubsection{ZkSync and Aztec}
ZkSync and Aztec are two Layer 2 networks that provide Ethereum scaling solutions and are already active on the Ethereum mainnet. To guarantee both privacy and scalability, they both use the Zero-Knowledge Rollup described in previous sections. Furthermore, while ZkSync seeks to be a generic protocol that could increase Ethereum's transaction throughput, Aztec prioritises privacy while retaining some scalability. The Rollup, according to the researchers, can handle 300 transactions per second while also allowing protected ERC-20 token transactions and private interactions with decentralised finance protocols. As part of a pooling contract, users would be able to trade on Uniswap and other exchanges.
\subsubsection{Plasma}
Plasma is a Layer 2 scaling solution focused on Ethereum's growth. After state channels, it is expected to be the second fully deployed scaling solution on the Ethereum mainnet. It is a platform that enables the establishment of child blockchains that rely on the Ethereum main chain for trust and arbitration. Child chains in Plasma can be developed to fulfil the needs of unique use cases, including those that are currently not possible on Ethereum. Plasma is better suited to decentralised apps that demand users to pay high transaction fees. However, the Plasma protocol is still in its early stages of development. The testing revealed a high throughput of up to 5,000 transactions per second, according to Ethereum specialists who participated. As a result, increasing the number of projects on the Ethereum Platform will not affect network congestion or transaction delays.
\subsubsection{Polygon - Matic}
Polygon (with the ticker MATIC) is a multi-chained system, a framework and a protocol all in one. It connects Ethereum-compatible blockchain networks, and as with other protocols we've talked about, it was created to address the present Ethereum network's scalability difficulties. Polygon uses side chains added to the primary platform cost-effectively and efficiently. Polygon's multi-chain network enables blockchain networks to communicate with one another outside of Ethereum's core chain while maintaining Ethereum's liquidity, security, and interoperability.

The Polygon ecosystem's core resource is MATIC, Polygon's token. In addition to being an asset, it is primarily utilised for staking tokens (proof-of-stake algorithm) to protect the Polygon network. The MATIC token has a maximum supply of 10 billion units, of which over 67\% are now in use. With a price of \$1.4 per token and a market capitalization of more than \$9 billion, it is currently among the top 25 cryptocurrencies in the world.
\subsubsection{StarkEx}
StarkEx is a Validium, which as mentioned in previous sections is a Layer-2 scaling solution that enforces the validity of all transactions using zero-knowledge proofs while keeping data availability off-chain. This prevents the Validium's funds from being stolen because any movement of value from a user's account must be authorised by that user.
For applications such as DeFi and gaming, StarkEx uses STARK technology to power scalable self-custodial transactions (trading and payments). StarkEx allows an application to scale up quickly and reduce transaction expenses. Moreover, StarkEx has a Verifier Contract Upgrade mechanism in place that allows the operator to immediately add a new item to the chain of verified contracts. You can't remove user signature checks, for example, because this would invalidate the old reasoning. Rather, it enables the inclusion of extra constraints.
\subsubsection{Hybrid Solutions}
Hybrid solutions incorporate the greatest features of different layer 2 technologies while also allowing for customisable trade-offs. One example of this is the Celer Network, which is utilised for off-chain scaling, cost-effective and quick dApp creation, blockchain-based games, and liquidity.


\subsection{Conclusion}

% why Refees is needed when all this pannel of layer 2 solutions is available in the market ?
