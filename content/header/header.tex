
    


\title{\projectName : a subscription-based model to hedge Gas Fees on the Ethereum Blockchain\\
%{\footnotesize \textsuperscript{*}Note: Sub-titles are not captured in Xplore and should not be used}
%\thanks{Identify applicable funding agency here. If none, delete this.}
}

\author{
\IEEEauthorblockN{Pau Autrand Caballero}
\IEEEauthorblockA{
\textit{EPFL}\\
Lausanne, Switzerland}
\and
\IEEEauthorblockN{Amin Debabeche}
\IEEEauthorblockA{
\textit{EPFL}\\
Lausanne, Switzerland}
\and
\IEEEauthorblockN{Lucas Giordano}
\IEEEauthorblockA{
\textit{EPFL}\\
Lausanne, Switzerland}
\and
\IEEEauthorblockN{Augustin Kapps}
\IEEEauthorblockA{
\textit{EPFL}\\
Lausanne, Switzerland}

}

\maketitle
\thispagestyle{plain}
\pagestyle{plain}

\begin{abstract}
Large gas prices and volatility are the cause of multiple problems on the Ethereum Mainnet. Several solutions such as Layer 2 or Gas Token were proposed over the last few years to deal with them. However, in the Mainnet, traffic keeps increasing and gas fees problems are far from being solved. To this end, we propose, \projectName, a scheme to harness the speculative power of gas fee volatility to a party, while at the same time offering insurance guarantees to another. 
This article is divided into two main sections. As a first step, we introduce the gas fee problem, investigate existing and future solutions to finally demonstrate the usefullness of our system. Secondly, we develop the mechanisms behind \projectName, as a proof of concept. 
\end{abstract}

\begin{IEEEkeywords}
Ethereum, gas fee, layer 2, subscription.
\end{IEEEkeywords}